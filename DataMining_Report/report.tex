% Created 2013-01-28 Mon 22:41
\documentclass[11pt]{article}
\usepackage{graphicx}
\usepackage{geometry}
\usepackage{amsmath}
\usepackage[pdftex]{hyperref}
\usepackage[font=small,labelfont=bf]{caption}
\geometry{a4paper, textwidth=6.5in, textheight=10in, marginparsep=7pt, marginparwidth=.6in}

\title{Data Mining 2013: Project Report}
\author{\textbf{Jared Niederhauser} - njared@student.ethz.ch\\
\textbf{Ruben Wolff} - wolffr@student.ethz.ch}
\date{\today}

\begin{document}
\maketitle

\section{Approximate retrieval - Locality Sensitive Hashing}
\begin{enumerate}
\item How was your choice of rows and bands motivated? How did you search for the
best parameters? \\ \\
\textbf{Answer}: \emph{We started with the formula in the book that related bands
to rows for a given similarity threshold value.  Next we started with the most
complex model we were allowed, (b*r $\approx$ 120). Finally we found the most
suitable value for b such that the value was still below the maximum allowed
value of 120}

\item Conceptually, what would you have to change if you were asked to design an image
  retrieval system that you can query for similar images given a large image
  dataset? \\

\textbf{Answer}: \emph{Depends on how the shingles are created}

\end{enumerate}

\section{Large-scale Supervised Learning}

\begin{enumerate}
\item Which algorithms did you consider? Which one did you choose for the
  final submission and why? \\ \\
\textbf{Answer}: \emph{Ruben to start}

\item How did you select the parameters for your model? Which are the
  most important parameters of your model? \\ \\
\textbf{Answer}: \emph{Ruben to start}

\end{enumerate}

\section{Recommender Systems}

\begin{enumerate}
\item Which algorithm did you implement? What was your motivation? \\ \\
\textbf{Answer}: \emph{TODO}

\item How did you select the parameters for your model? \\ \\
\textbf{Answer}: \emph{TODO}

\item Does the performance measured in CTR increase monotonically during the
execution of your algorithm? Why? \\ \\
\textbf{Answer}: \emph{TODO}

\end{enumerate}

\end{document}